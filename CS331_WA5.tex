\documentclass{article}
\usepackage[utf8]{inputenc}
\usepackage{amsmath}
\usepackage[tmargin=1in,bmargin=1in,lmargin=1.25in,rmargin=1.25in]{geometry}
\usepackage{graphicx}



\begin{document}


%%%%%%%%%%%%%%%%%%%%%%%%%%%%%%
% Document Header
%%%%%%%%%%%%%%%%%%%%%%%%%%%%%%
\begin{center}
    \Large{\textbf{CS331 (Spring 2022): Introduction to Artificial Intelligence}}

   \Large{\textbf{ Written Assignment \#5}}
   
   \vspace{3mm}
   
   Andrea Tongsak \\ tongsaka@oregonstate.edu
\end{center}

\vspace{5mm} 

\noindent Date handed out: May 18, 2022 \\
\noindent Date due: May 25, 2022, 10am, on Canvas\\
\noindent Total: 25 points

\vspace{5mm} 

\noindent This assignment is to be done individually. Please hand in a pdf on Canvas. 

\vspace{5mm} 


%%%%%%%%%%%%%%%%%%%%%
% Problem 1
%%%%%%%%%%%%%%%%%%%%

\noindent 1.  Consider the game represented in normal form below. 


\vspace{3mm}

\begin{table*}[h]
    \centering
    \begin{tabular}{|c|c|c|c|}
    \hline
    & B: S1  & B: S2 & B: S3 \\
    \hline
    A: S1 & A = 5, B = 9 & A = 3, B = 6 & A = 1, B = -4 \\
    \hline
    A: S2 & A = 4, B = -4 &  A = 2, B = 2  & A = -1, B = 6 \\
    \hline
    A: S3  & A = 0, B = 0  & A = 2, B = -2  & A = 0, B = 0 \\
    \hline
    \end{tabular}
\end{table*}
\vspace{3mm}

\noindent a)  Indicate all the pure strategy Nash equilibria in the matrix. \textbf{[2 points]}

\vspace{3mm}

\noindent \textbf{ANSWER:}

%%%%%%%%%%%%%%%%%%%%%%%
%
% Your answer to 1 a)
%
%%%%%%%%%%%%%%%%%%%%%%

My responses will be denoted (A's strategy, B's strategy).

The pure strategy Nash equilibriums in the matrix are (S1,S1).

In general, a strategy profile forms an equilibrium if no player can benefit by switching strategies, given that every player sticks with the same strategy.

A Nash equilibrium is a strategy profile in which no player can benefit by switching strategies, given that the other's players remain unchanged. Pure strategy is a deterministic policy, and will NOT always mean it is the best. Instead, they want to maximize their own utility.

\vspace{5mm}

\noindent b) Does Player A have a strictly dominant strategy? If yes, state what it is. If no, explain 
why not. \textbf{[2 points]}

\vspace{3mm}

\noindent \textbf{ANSWER:}

%%%%%%%%%%%%%%%%%%%%%%%
%
% Your answer to 1 b)
%
%%%%%%%%%%%%%%%%%%%%%%

Yes. S1 is the strictly dominant strategy for player A. 

For player A, choosing strategy S1 would give them the highest payoffs. S1 strictly dominates S2 because S1 would always give a better outcome than choosing S2 (no matter what the other player does). S1 also strictly dominates S3 since S1 will again give the better outcome.

\vspace{5mm}

\noindent c) Does Player B have a strictly dominant strategy? If yes, state what it is. If no, explain 
why not.\textbf{ [2 points] }

\vspace{3mm}

\noindent \textbf{ANSWER:}

%%%%%%%%%%%%%%%%%%%%%%%
%
% Your answer to 1 c)
%
%%%%%%%%%%%%%%%%%%%%%%

No, player B does not have a strictly dominant strategy. 

Player B has a weakly dominant strategy. For player B, S1 is weakly dominated by S3. The one difference is that in S1, there is an outcome that gives B=9 instead of B=6. Other than that, the rest of S1 and S3's results are similar, e.g. B=-4, B=0

\vspace{5mm}

\noindent d) What is the Pareto optimal outcome in this game? \textbf{[2 points]} 

\vspace{3mm}

\noindent \textbf{ANSWER:}

%%%%%%%%%%%%%%%%%%%%%%%
%
% Your answer to 1 d)
%
%%%%%%%%%%%%%%%%%%%%%%
The Pareto optimal outcome in this game is (S1, S1).

A Pareto optimal outcome is one where there is no other outcome that all players would prefer.


\vspace{5mm}

\noindent e) Is this a zero-sum game? \textbf{[2 points] }

\vspace{3mm}

\noindent \textbf{ANSWER:}

%%%%%%%%%%%%%%%%%%%%%%%
%
% Your answer to 1 e)
%
%%%%%%%%%%%%%%%%%%%%%%

This game is not zero-sum. 

Zero-sum games are defined as games where the total payoff to all players is the same for every instance of the game. Zero-sum games are also games in which the sum of the payoffs is always zero.

There are some outcomes where the payoffs between the players are no longer zero, for instance, (S1, S1), (S1, S2), (S1, S3), and (S2, S3). These outcomes favor B. 



\vspace{10mm}

\clearpage

%%%%%%%%%%%%%%%%%%%%%
% Problem 2
%%%%%%%%%%%%%%%%%%%%

\noindent 2.  Consider the game represented in normal form below. 

\vspace{3mm}



\begin{table*}[h]
    \centering
    \begin{tabular}{|c|c|c|}
    \hline
       & B: S1 & B: S2 \\ 
       \hline
        A: S1  & A = 2, B = -2 & A = -5, B = 5\\ 
        \hline
        A: S2 & A = -3, B = 3 & A = 4, B = -4 \\
        \hline
    \end{tabular}
\end{table*}

\vspace{5mm}

\noindent a) Calculate the mixed strategy Nash equilibrium for this game. Clearly indicate the 
probability associated with each strategy for each player. \textbf{[10 points] }

\vspace{3mm}

\noindent \textbf{ANSWER:}

%%%%%%%%%%%%%%%%%%%%%%%
%
% Your answer to 2 a)
%
%%%%%%%%%%%%%%%%%%%%%%

A mixed strategy is a randomized policy, based on selecting a strategy based on a probability distribution.

\textbf{\underline{Mixed strategy for A:}}

[S1: q, S2: 1-q]\\

[S1: $\frac{1}{2}$, S2: $\frac{1}{2}$]\\


\textbf{B's payoff for S1?}
\begin{align}
-2q + 3(1-q) = -2q + 3 - 3q = -5q + 3
\end{align}



\textbf{B's payoff for S2?}
\begin{align}
5q - 4(1-q) = 5q - 4 + 4q = 9q - 4
\end{align}


\textbf{Set equal and solve for q:}
\begin{align}
-5q + 3 &= 9q - 4\\
7 &= 14q\\
q &= \frac{1}{2}
\end{align}



\textbf{\underline{Mixed strategy for B:}}

[S1: p, S2: 1-p]\\

[S1: $\frac{9}{14}$, S2: $\frac{5}{14}$]\\

\textbf{Suppose A: S1}
\begin{align}
2p - 5(1-p) = 2p - 5 + 5p = 7p - 5
\end{align}


\textbf{Suppose A: S2}
\begin{align}
-3p + 4(1-p) = -3p + 4 - 4p = -7p + 4
\end{align}

\textbf{Set equal and solve for p:}
\begin{align}
7p - 5 &= -7p + 4\\
14p &= 9\\
p &= \frac{9}{14}
\end{align}

\vspace{5mm}

\noindent b)  Calculate the expected payoffs for each player at the mixed strategy Nash equilibrium 
you calculated. \textbf{[5 points]}

\vspace{3mm}

\noindent \textbf{ANSWER:}

%%%%%%%%%%%%%%%%%%%%%%%
%
% Your answer to 2 b)
%
%%%%%%%%%%%%%%%%%%%%%%

Player B:
Computed expected payoff for B:
\begin{align}
9(\frac{1}{2}) - 4 &= \frac{1}{2}
\end{align}

Player A:
Computed expected payoff for A:
\begin{align}
7(\frac{9}{14}) - 5 &= -\frac{1}{2}	
\end{align}







\end{document}
